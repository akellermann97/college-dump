\documentclass[12pt]{article}
\usepackage{mathtools}
\usepackage{amsfonts}
\usepackage{tikz}


\usetikzlibrary{chains, positioning, shapes, arrows, matrix, calc, fit}
\tikzset{circ/.style={draw,circle,node distance=2mm,inner sep=0.1pt},topath/.style={to path={|-(\tikztotarget)}}}




\DeclarePairedDelimiter\ceil{\lceil}{\rceil}
\DeclarePairedDelimiter\floor{\lfloor}{\rfloor}


\setlength{\oddsidemargin}{0in}
\setlength{\evensidemargin}{\oddsidemargin}
\setlength{\textwidth}{6.5in}
\setlength{\textheight}{9in}
\setlength{\topmargin}{-0.25in}
\newcommand{\bb}{$\Box$}
\newcommand{\sline} {$\mbox{\underline{\hspace{4cm}}}$}
\newcommand{\lline} {$\mbox{\underline{\hspace{4cm}}}$}
\newcommand{\llline} {$\mbox{\underline{\hspace{8cm}}}$}

\newcommand{\double}{\mbox{\em double}}
\pagestyle{empty}

\begin{document}
\begin{center}
{\bf 
Introduction to Cryptography (462) \\
Homework 07\\
T.J. Borrelli\\
Due: Thursday, November 30th, 2017 at 2pm
}

\end{center}



\begin{itemize}
\item Be sure to put your NAME and Section number on the first page. 
\item If you upload your submission to the myCourses dropbox, I will only accept .pdf format and only the last thing you submit will be accepted. 
\item This homework is related to Chapter 7 in the Paar and Pelzl (P\&P) book and notes. 
\item This is the last graded hw. 
\item {\bf For each question, show the details of your computation unless otherwise specified. } 
\end{itemize}





\begin{enumerate}





\vspace{.25cm}

% Q1 - RSA
\item
Encrypt and decrypt by means of the RSA algorithm with the following system parameters: 

\begin{enumerate}

\item
$p=3, q=11, d=7, x = 5$ 

\item
$p=5, q=11, e=3, x=9$ \\
\end{enumerate}


\vspace{.5cm}

%Q2 - CRT
\item
Consider moduli 11 and 13 in the Chinese Remainder Theorem. What numbers are represented by the pairs (1,0), (4,5) and (5,4)? Show the details of your work.

\vspace{.5cm}

% Q3 - CRT
\item
Find two non-standard roots (not 1, nor -1) of $\sqrt{1}$ in $\mathbb{Z}_{77}$.

 %$\mathbb{Z}_{m}$.

\vspace{.5cm}


% Q4 - primality
\item
Use your favorite programming language to implement the Fermat Primality Test.  
(Note: your program should use the Square-and-Multiply algorithm from last time.) \\

Use your program to find  the last three Carmichael numbers less than $10^{6}$ and
the last three Carmichael numbers less than $10^{7}$.  \\

Submit your code in the PDF file as usual. 


\end{enumerate}
\end{document}
